% !TEX root = main.tex

\section{A Case Study}
\label{sec.cash}

This section shows how the number of symbolic states can be
dramatically reduced by slightly introducing guarded terms in the
existing symbolic specification of CASH
in~\cite{rocha-rewsmtjlamp-2017}. 
%
The CASH algorithm was first formally specified and analyzed in
Real-Time Maude using explicit-state methods
\cite{olveczky2006formal}, and symbolically analyzed using
rewriting modulo SMT to deal with an infinite number of states
\cite{rocha-rewsmtjlamp-2017}. 
%
Due to the symbolic state space explosion problem and the
complexity of the accumulated constraints,
some reachability
properties of interest were previously beyond the reach of rewriting modulo
SMT in \cite{rocha-rewsmtjlamp-2017},
but can now be analyzed using guarded terms.


\subsection{Symbolic States Using Guarded Terms}


In CASH, each task has a given period and a given worst-case execution
time.  When an instance of a task is completed before its worst-case
execution time, the unused processor time is added to a global queue
of unused budgets.  Another task can then use these unused execution
times as well as its own execution budget, instead of just wasting
them.  In addition to capacity sharing, tasks are scheduled according
to standard preemptive EDF scheduling: the task with the shortest time
remaining until its deadline has the priority and can preempt the
currently executing task.  The reader is referred
to~\cite{caccamo2000capacity,olveczky2006formal} for details about the
CASH algorithm.



In rewriting logic specification, an object of class $C$ is
represented as a term \texttt{<~O : C |
  $\mathit{att}_1$\;:\;$v_1$,\;\ldots,\;$\mathit{att}_n$\;:\;$v_n$~>}
of sort \sort{Object}, where $O$ is the identifier, and $v_1, \ldots,
v_n$ are the values of the attributes $\mathit{att}_1, \ldots,
\mathit{att}_n$ \cite{clavel-maudebook-2007}.  A system state, called
a \emph{configuration}, is declared as a multiset of these objects.
Multiset union for configurations is given by a juxtaposition operator
that is associative and commutative and having the \texttt{none}
multiset as its identity element.  

In the CASH specification, the
configuration consists of a number of servers (i.e., tasks) and a
global queue. % of unused budgets.  
A server state consists of six
attributes: the maximum budget, period, internal state (\texttt{idle},
\texttt{waiting}, or \texttt{executing}), time executed, budget time
used, and time to deadline.  These variables are modeled as
nonnegative integers. Specifically, for internal states, $0$ denotes
\texttt{idle}, $1$ denotes \texttt{waiting}, and $2$ denotes
\texttt{executing}.  A server object with name $O$ is modeled as a
term of sort \sort{Object} and has the form

\begin{lstlisting}[mathescape,basicstyle=\ttfamily\footnotesize,columns=flexible]
< $O$ : server | maxBudget : $m$,          period : $p$,          state : $s$, 
                 timeExecuted : $e$, usedOfBudget : $u$, timeToDeadline : $d$ >
\end{lstlisting}


A global object contains the global queue of unused budgets, the
availability of the processor, and a flag to denote whether any
deadline has been missed.  Both availability and deadline missed flags
are modeled as Boolean values.  A global object with name $G$ is
modeled as a term of sort \sort{Object} and has the form

\begin{lstlisting}[mathescape,basicstyle=\ttfamily\footnotesize,columns=flexible]
< $G$ : global | cq : $\mathit{queue}$,  available : $a$, deadline-miss : $b$ >
\end{lstlisting}


A global queue is a priority queue of unused budgets.  Each item in a
queue is a pair \texttt{(deadline: $i$ budget: $i'$)} of deadline $i$
and budget $i'$.  An item with the smallest deadline has the highest
priority.  In \cite{olveczky2006formal,rocha-rewsmtjlamp-2017}, a
global queue is specified as a sorted list.  Instead, in this work the
queue is modeled as a \emph{multiset} of items in order to
symbolically define queue operations using only SMT constraints.  For
example, consider a queue \[\mathit{cq} \;\equiv\;
(\texttt{deadline:}\; i_1 \texttt{ budget:}\; i_1') \;\ldots\;
(\texttt{deadline:}\; i_n \texttt{ budget:}\; i_n')\] with $n$ items.
The operation to peek the highest-priority element is defined by:
\[
\mathit{peek}(\mathit{cq}) = (\texttt{deadline:}\; i_k \texttt{ budget:}\; i_k')
\;\iff\;
\textstyle
\bigwedge_{j =1}^n i_k \leq i_j.
\]

Guarded terms are used in the new symbolic specification in a number
of ways. First, each variable in a server state or a global object can
be guarded by constraints.  For example, the guarded term $0_{\phi}
\vee 1_{\neg \phi}$ denotes an internal state of a server which is
either $0$, if the condition $\phi$ holds, or $1$, otherwise.  Second,
each item in a server state can be guarded by constraints.  In a
similar way to the example in Section~\ref{sec.rewsmt}, the guarded
term $(\texttt{deadline:}\; i \texttt{ budget:}\; i') |_\phi \vee
\mathit{emptyQueue} |_{\neg \phi}$ denotes an item which is either
present if the condition $\phi$ is satisfiable, or absent otherwise.

Some syntactic sugar is defined to succinctly write these guarded
terms as follows: $\phi\ ?\ i : j$ denotes the guarded term $i_{\phi}
\vee j_{\neg \phi}$, for $i, j \in \mathbb{Z}$, and $(e \texttt{
  const:}\; \phi)$ denotes the guarded term $e |_\phi \vee e |_{\neg
  \phi}$, for each item $e$ in the priority queue.  For example, the
following expression includes both types of guarded terms:

\begin{maude}
(deadline: I3 >= 1 ? I3 - 1 : 0  budget: I2 >= 1 ? 1 : 0  const: I2 > 1).
\end{maude}

\subsection{Symbolic Transitions Using Guarded Rewrite Rules}


The symbolic transitions of the CASH algorithm are
specified by $13$ conditional rewrite rules in
\cite{rocha-rewsmtjlamp-2017}, which are adapted from the Real-Time
Maude specification in \cite{olveczky2006formal}.  The rule conditions
specify constraints solvable by the SMT decision procedure, together
with some extra conditions to generate constraints by rewriting.
% to reduce the size of the specification.
%This section explains how to remove symbolic branching using guarded terms. 
This section highlights some of the rules to explain how to remove
symbolic branching using guarded terms.  The reader is referred
to  \cite{bae-noteguardedterms-2017} for the full
description of all (guarded) rewrite rules.


\paragraph{Rule \texttt{stopExecuting1}.}

This rule specifies when one server completes execution while another
server is waiting.  The waiting server with the least time remaining
until its deadline starts executing, and any budget left is added to
the global queue.  Previously, this behavior was specified by two
rewrite rules in \cite{rocha-rewsmtjlamp-2017}, depending on whether
any budget remains.  However, with guarded terms only one rewrite rule
is needed:


\begin{maude}
   {< G  : global | cq : CQ, AtSG >
    < O  : server | state : St, 		usedOfBudget : T, 
                    maxBudget : NZT,		timeToDeadline : T1, 
                    timeExecuted : NZT1, 	period : NZT2, AtS >
    < O' : server | state : St', timeToDeadline : T2, AtS' > REST}
=> {< G  : global | cq : (deadline: T1  budget: (NZT monus T)  
                          const: (NZT > T)) CQ, AtSG >
    < O  : server | state : idle, 		usedOfBudget : NZT, 
                    maxBudget : NZT, 		timeToDeadline : T1, 
                    timeExecuted : NZT1, 	period : NZT2, AtS >
    < O' : server | state : executing, timeToDeadline : T2, AtS' > REST}
if ALL�\footnote{The specification of \texttt{ALL} has been omitted in the rule. \texttt{ALL}  represents
the entire configuration in the left hand side.}� := ...  /\  (St === executing) /\  (St' === waiting)  /\  (NZT > 0)
/\ (NZT1 > 0)  /\  (NZT2 > 0)  /\  (T >= 0)  /\  ((NZT monus T) <= T1)�\footnote{The \texttt{monus} operator is defined by: $a \texttt{ monus } b = \max(a-b, 0)$.}�
/\ (T1 >= 0)  /\  (T2 >= 0)  /\  nextDeadlineWaiting(ALL,O,T2)�\footnote{The function \texttt{nextDeadlineWaiting(ALL,O,T2)} returns a constraint stating that the \texttt{timeToDeadline}
of all servers, except \texttt{O}, is at least \texttt{T2}.}�
\end{maude}

The guarded term
%
\lstinline[basicstyle=\ttfamily\footnotesize,columns=flexible]{(deadline: T1 budget: (NZT monus T) const: (NZT > T))},
%
which is only present
in the priority queue if the condition
%
\lstinline[basicstyle=\ttfamily\footnotesize,columns=flexible]{NZT > T}
%
is satisfiable, is written in the \texttt{global} object in the right
hand side of the rewrite rule.  Unlike~\cite{rocha-rewsmtjlamp-2017},
no constraints are yet added regarding the ``position'' of the newly
added item in the queue, because:
\begin{inparaenum}[(i)]
	\item the item may be absent if its constraint is not satisfiable, and
	\item the priority queue is specified as a multiset with symbolic
    constraints.
\end{inparaenum}

\paragraph{Rule \texttt{tickExecutingSpareCapacity}.}

This rule models time elapse when a server is executing a spare
capacity, specified as one guarded rewrite rule below.%
\footnote{The functions \texttt{delta-global} and \texttt{delta-servers} model the time lapse;
that is, variables in servers and the queue are accordingly increased or decreased (by $1$)
based on the current state.}
The left hand side of the rule contains the guarded term 
\lstinline[basicstyle=\ttfamily\footnotesize,columns=flexible]{(deadline: I1 budget: I2 const: iB)}.
The condition \texttt{iB} in the rule condition specifies that the item is present in the queue.
The condition \lstinline[basicstyle=\ttfamily\footnotesize,columns=flexible]{aboveOrEqualDeadline(I1,CQ)}
states that all deadlines in \texttt{CQ} are at least \texttt{I1},
that is, that the item is indeed the one with the highest priority.


\begin{maude}
   {< G : global | cq : (deadline: I1 budget: I2 const: iB) CQ, AtSG >
    < O : server | state : St, 
                   timeExecuted : T1, 	timeToDeadline : T2, AtS > REST}
=> { delta-global( < G : global | cq : (deadline: I1 budget: (I2 + (- 1)) 
                     const: (iB and (I2 > 1))) CQ, AtSG >, 1)
     delta-servers(< O : server | state : executing, 	timeExecuted : T1,
                     timeToDeadline : T2, AtS > REST, 1, false)}
if ALL := ...  /\  (St === executing)  /\  (T1 >= 0) /\  (T2 >= 0) 
/\ iB  /\  (I1 >= 1)  /\  (I2 >= 1)  /\  (T2 >= 1)  /\  (I1 <= T2)
/\ mte-server(ALL,O,1)  /\  noDeadlineMiss(ALL)
/\ aboveOrEqualDeadline(I1,CQ)
\end{maude}


\subsection{Symbolic Reachability Analysis}

The goal is to symbolically analyze if missed deadlines exist for %(a
variant of) the CASH algorithm from an infinite set of initial
configurations containing two servers $s_1$ and $s_2$.  Suppose that
server $s_i$ (for $i = 1, 2$) has maximum budget $b_i$ and period
$p_i$ such that $0 \leq b_i < p_i$.  No deadline misses are guaranteed
only when the processor utilization is less than or equal to 100\%
(i.e., $b_1 / p_1 + b_2 / p_2 \leq 1$) \cite{caccamo2000capacity}.  A
counterexample exists if there is no constraint on the initial
configurations.  Indeed, the previous work
\cite{rocha-rewsmtjlamp-2017} can find a symbolic counterexample that
violates this constraint using rewriting module SMT.

A more interesting problem is to check the existence of missed
deadlines of the \emph{variant} of the CASH algorithm
\cite{olveczky2006formal}, that uses a different scheduling strategy,
when $b_1 / p_1 + b_2 / p_2 \leq 1$ is satisfied.  Using
explicit-state model checking, it is shown
in~\cite{olveczky2006formal} that the variant can have a
counterexample, even though $b_1 / p_1 + b_2 / p_2 \leq 1$ is
satisfied.  Since this constraint is non-linear, it is beyond the
capabilities of state-of-the-art SMT solvers.  But by fixing $p_1$ and
$p_2$, say $p_1 = 5$ and $p_2 = 7$, the constraint becomes linear and
symbolic analysis can be performed using existing SMT solvers.

Here an infinite set of initial configurations with two servers $s_1$
and $s_2$ is considered.  Server $s_i$ (for $i = 1, 2$) has maximum
budget $b_i$ and period $p_i$ such that $0 \leq b_i < p_i$, and has
the constraint $b_1 / p_1 + b_2 / p_2 \leq 1$.  A symbolic state can
be modeled as term
\lstinline[basicstyle=\ttfamily\footnotesize,columns=flexible,mathescape]{init($b_1$,$p_1$,$b_2$,$p_2$)}.
%
The previous work in~\cite{rocha-rewsmtjlamp-2017} cannot deal with
this problem due to the symbolic state space explosion; it generates
nearly one billion symbolic states and does not terminate for a few
days.  On the other hand, the new specification using guarded terms
can successfully find a counterexample from
\lstinline[basicstyle=\ttfamily\footnotesize,columns=flexible,mathescape]{init($b_1$,$5$,$b_2$,$7$)}:

\begin{maude}
search [1] init(I0,5,I2,7) 
       =>* {B:Bool , < g : global | deadline-miss : true, AtS > Cnf} .
Solution 1 (state 590751)
states: 590752  rewrites: 1910935222 in 5513748ms cpu (5513848ms real) 
B --> not I0 < 0 and not I2 + -5 < 0 and not 1 + - I0 < 0 and not 7 + - 
 I2 < 0 and not 35 + - (I2 * 5) + - (I0 * 7) < 0 and (not 7 + - I2 < 0 or 
 I2 + -1 < 0) and (not 12 + - I2 < 0 or I2 < 0) and (not 17 + - I2 < 0 or 
 I2 < 0) and - I0 < 0 and I2 + -7 < 0 and I0 + -5 < 0 and 4 + - I2 < 0
Cnf --> < s1 : server | maxBudget : I0, period : 5, state : 2,
              usedOfBudget : 0, timeToDeadline : 0, timeExecuted : 5 > 
        < s2 : server | maxBudget : I2, period : 7, state : 0,
              usedOfBudget : I2, timeToDeadline : 7, timeExecuted : 2 >
AtS --> cq : deadline: 7 budget: not I2 < 0 ? I2 + -5 : -5 
             const: (not I2 < 0 and 5 + - I2 < 0), available : false
\end{maude}

\noindent The counterexample is found at search depth $23$, which
could not be reached without guarded terms.  A sequence of symbolic
rewrites to reach this counterexample can be obtained by Maude's
\texttt{show path} command.
%
In the counterexample, \texttt{B} represents the accumulated constraint
whose satisfiable assignment gives a concrete counterexample. 
The queue \texttt{cq} has the single item 
if the constraint \lstinline[basicstyle=\ttfamily\footnotesize,columns=flexible]{(not I2 < 0 and 5 + - I2 < 0)} is satisfied; otherwise, \texttt{cq} is empty.
%
For example, \texttt{B} has the satisfiable assignment $\texttt{I0}
=1$ and $\texttt{I2} =5$, which is found by an SMT solver, and in this
case the queue \texttt{cq} is empty.  An explicit search with these
concrete values in the \textit{ground} rewriting logic semantics of
CASH gives an expected result as follows:

\begin{maude}
search [1] init(1,5,5,7) 
       =>* {true, < g : global | deadline-miss : true, AtS > Cnf} .
Solution 1 (state 107393)
states: 107394  rewrites: 36507422 in 29543ms cpu (29699ms real) 
Cnf --> < s1 : server | maxBudget : 1, period : 5, state : 2, 
              usedOfBudget : 0, timeToDeadline : 0, timeExecuted : 5 > 
        < s2 : server | maxBudget : 5, period : 7, state : 0, 
              usedOfBudget : 5, timeToDeadline : 7, timeExecuted : 2 >
AtS --> cq : emptyQueue, available : false
\end{maude}
