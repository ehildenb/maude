% !TEX root = main.tex

\section{Related Work and Concluding Remarks}
\label{sec.concl}

SMT-based reachability analysis has been used in software component
testing and verification of infinite-state
systems~\cite{cadar-seswtesting2013}.
SMT-CBMC~\cite{armando2009bounded} and Corral~\cite{corral} use
bounded model checking, with unbounded types represented by built-in
variables. KLEE~\cite{klee} is used for symbolic execution and
constraint solving, finding possible inputs that will cause a
programming artifact to crash.  The {IC3} SMT-based model
checker~\cite{cimatti-ic32012} uses over- and under-approximation
techniques to efficiently handle symbolic transitions on infinite sets
of states. What all these approaches have in common is the effort in
developing advanced techniques and tools to speed up the reachability
analysis process based on SMT-solving. Guarded terms are a technique
to speed up and often attain convergence of the reachability analysis
process for rewriting modulo SMT.
%
It complements
narrowing-based reachability
analysis~\cite{abstract-lmc-bae,meseguer-narrowing-2007}, another
symbolic technique combining narrowing modulo theories and model
checking.

This paper presented \textit{guarded terms}, a technique with the
potential to reduce the symbolic state space and the complexity of
constraints in the rewriting modulo SMT approach. Rewriting modulo
SMT~\cite{rocha-rewsmtjlamp-2017} is a novel symbolic technique to
model and analyze reachability properties of infinite-state
systems. This is a technique in the realm of rewriting logic that can
greatly improve the specification and verification of reachability
properties of open systems such as real-time and cyber-physical
systems. Guarded terms generalize the constrained terms of rewriting
modulo SMT by allowing a term in a symbolic state to have constrained
subterms: these subterms can be seen as a choice operator that is part
of the term structure. They can be composed in parallel and be nested,
thus enabling the succinct encoding of several constrained terms into
one guarded term.  The potential of guarded terms for reducing the
symbolic state-space, and the complexity and size of constraints has
been illustrated by a running example and a case study. The latter is
an improvement of a previously developed case study where guarded
terms enable the analysis of reachability properties of the CASH
scheduling algorithm~\cite{caccamo2000capacity}.

As future work, the plan is to explore the use of guarded terms in
improving the symbolic rewriting logic semantics of
PLEXIL~\cite{rocha-thesis-2012,rocha-rewsmtjlamp-2017}, and in
specifying the symbolic rewriting logic semantics of other real-time
and cyber-physical systems. Other SMT techniques, including state
subsumption, backwards reachability, $k$-induction, and interpolants,
should certainly be studied for rewriting modulo SMT. Another
perspective is to use guarded terms for improving narrowing-based
reachability analysis.

{\small \paragraph*{\bf Acknowledgments.}

The first author was supported by the Basic Science Research Program
through the National Research Foundation of Korea (NRF) funded by the
Ministry of Education (2016R1D1A1B03935275). The second author has
been supported in part by the EPIC project funded by the
Administrative Department of Science, Technology and Innovation of
Colombia (Colciencias) under contract 233-2017.}
