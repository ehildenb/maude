\documentclass[11pt]{beamer}

\usetheme{metropolis}
\usepackage{appendixnumberbeamer}
\usepackage{amsmath,amssymb,stmaryrd,mathtools}
\usepackage{cite}
\usepackage{color,listings}
\usepackage{enumerate}
\usepackage{fancyvrb}
\usepackage[utf8]{inputenc}
\usepackage{paralist}
\usepackage{pgfplots}
\usepgfplotslibrary{dateplot}
\usepackage{thmtools,thm-restate}
\usepackage{tikz}
\usepackage[normalem]{ulem}   %%% CAM
\usepackage{url}
\usepackage{xspace}
\usepackage[all]{xy}
\usetikzlibrary{arrows,shapes,decorations.pathmorphing,backgrounds,positioning,fit,petri}

% misc
\newcommand{\ign}[1]{}

% sub
\newcommand{\usub}[1]{\raisebox{-0.5ex}{\ensuremath{\scriptstyle #1}}}

% sets
\newcommand{\nats}{\mathbb{N}}

% auxiliary notation and functions for equations and rules
\newcommand{\EQ}{=}
\newcommand{\IFS}{\mathbf{if}}
\newcommand{\REW}{\rightarrow}
\newcommand{\eq}[2]{#1 \EQ #2}
\newcommand{\ceq}[3]{#1 \EQ #2 \; \IFS\; #3}
\newcommand{\rl}[2]{#1 \REW #2}
\newcommand{\crl}[3]{#1 \REW #2 \; \IFS\; #3}
\newcommand{\lbl}[1]{[\textit{#1}]}
\newcommand{\cond}{\gamma}
\newcommand{\codd}[1]{\textit{#1}}
\newcommand{\RL}[1]{\ensuremath{[\mathit{#1}]}}

% symbolic relation and transitions
\newcommand{\cntr}[1]{\left\llbracket #1 \right\rrbracket}
\newcommand{\crel}{\rightsquigarrow}
\newcommand{\ecrel}{\rightrightarrows}
\newcommand{\lccrl}[4]{#1 : #2 \rel #3 \left\llbracket #4 \right\rrbracket}
\newcommand{\ccrl}[3]{#1 \REW #2\;\textnormal{\bf if}\; {#3}}
\newcommand{\uccrl}[2]{#1 \REW #2}
\newcommand{\ecrl}[4]{#1 \Rel #2 \cntr{#3} \;\textnormal{\bf if}\; {#4}}
\newcommand{\sccrl}[4]{#1 \rel^{#4} #2 \left\llbracket #3 \right\rrbracket}
\newcommand{\nlccrl}[4]{#1 : #2 \not\rel #3 \left\llbracket #4 \right\rrbracket}
\newcommand{\nccrl}[3]{#1 \not\rel #2 \left\llbracket #3 \right\rrbracket}
\newcommand{\srews}{\rightsquigarrow}
\newcommand{\can}[2]{{#1}\!\downarrow_{#2}}
\newcommand{\cans}[1]{{#1}\!\downarrow}

\newcommand{\Univ}{\mathcal{U}}
\newcommand{\rel}{\rightarrow}
\newcommand{\erel}{\twoheadrightarrow}
\newcommand{\Rel}{\Rightarrow}
\newcommand{\pair}[2]{\left( #1 \,; #2 \right)}
\newcommand{\triple}[3]{\left( #1 \,; #2 \,; #3 \right)}
\newcommand{\ppair}[2]{\langle #1 ; #2 \rangle}
\newcommand{\normrel}{\rel^\downarrow}
\newcommand{\arel}{\rel^\diamond}
\newcommand{\prel}{\rel^\Vert}
\newcommand{\abst}{T}
\newcommand{\paref}[1]{\stackrel{#1}{\rel_{\flat}}}
\newcommand{\parel}[1]{\stackrel{#1}{\rel_{\lcal}}}
\newcommand{\Parel}[1]{\stackrel{#1}{\rel}}

% FOL
%\newcommand{\lang}{\Lambda}
\newcommand{\lang}{{0}}
\newcommand{\nlang}{{1}}
\newcommand{\modeln}{M}
\newcommand{\model}{\tcal_{\ecal_\lang}}
\newcommand{\domain}{\modeln}
\newcommand{\inter}{\__\model}
\newcommand{\fvars}[1]{\textit{fvars}(#1)}
\newcommand{\strue}{\top}
\newcommand{\sfalse}{\bot}

% calligraphic 
\newcommand{\acal}{\mathcal{A}}
\newcommand{\bcal}{\mathcal{B}}
\newcommand{\ccal}{\mathcal{C}}
\newcommand{\dcal}{\mathcal{D}}
\newcommand{\ecal}{\mathcal{E}}
\newcommand{\fcal}{\mathcal{F}}
\newcommand{\gcal}{\mathcal{G}}
\newcommand{\hcal}{\mathcal{H}}
\newcommand{\ical}{\mathcal{I}}
\newcommand{\jcal}{\mathcal{J}}
\newcommand{\kcal}{\mathcal{K}}
\newcommand{\lcal}{\mathcal{L}}
\newcommand{\mcal}{\mathcal{M}}
\newcommand{\ncal}{\mathcal{N}}
\newcommand{\ocal}{\mathcal{O}}
\newcommand{\pcal}{\mathcal{P}}
\newcommand{\qcal}{\mathcal{Q}}
\newcommand{\rcal}{\mathcal{R}}
\newcommand{\scal}{\mathcal{S}}
\newcommand{\tcal}{\mathcal{T}}
\newcommand{\ucal}{\mathcal{U}}
\newcommand{\vcal}{\mathcal{V}}
\newcommand{\wcal}{\mathcal{W}}
\newcommand{\xcal}{\mathcal{X}}
\newcommand{\ycal}{\mathcal{Y}}
\newcommand{\zcal}{\mathcal{Z}}

% auxiliary notation and functions for sorts and terms
\newcommand{\sort}[1]{\textit{#1}}
\newcommand{\func}[3]{#1 : #2 \longrightarrow #3}
\newcommand{\tops}[1]{\left[#1\right]}
\newcommand{\dom}[1]{{\it dom}(#1)}
\newcommand{\ran}[1]{{\it ran}(#1)}
\newcommand{\vars}[1]{\textit{vars}(#1)}
\newcommand{\ids}{\textit{id}}
\newcommand{\CSU}{\textit{CSU}}
\newcommand{\GU}{\textit{GU}}
\newcommand{\csu}[3]{\CSU_{#1}(#2 \EQ #3)}
\newcommand{\gsu}[3]{\GU_{#1}(#2 \EQ #3)}
\newcommand{\ls}[1]{\mathit{ls}(#1)}
\newcommand{\sord}[1]{\ll_{#1}}
\newcommand{\posrm}{{\it pos}}
\newcommand{\bool}{\sort{Bool}}
\newcommand{\boolq}{[\bool]}
\newcommand{\true}{\top}
\newcommand{\false}{\bot}
\newcommand{\states}{\sort{State}}
%\newcommand{\states}{\mathfrak{s}}
\newcommand{\rews}{\rightarrow}
\newcommand{\rewsn}[1]{\rew^{#1}}
\newcommand{\Pip}{\Pi_{s_1,\ldots,s_n}}
%% \newcommand{\vpredp}{\vec{x}}
%% \newcommand{\vpredi}{\vec{y}}
\newcommand{\always}{\square}
\newcommand{\onext}{\bigcirc}
\newcommand{\imps}{\Rightarrow}
\newcommand{\cons}{\Longleftarrow}
\newcommand{\dens}[2]{\llbracket #1 \rrbracket_{#2}}
\newcommand{\den}[2]{\left\llbracket #1 \right\rrbracket_{#2}}
\newcommand{\olist}[1]{\mathbf{#1}}
\newcommand{\meta}[1]{\overline{#1}}
\newcommand{\unc}[1]{\textbf{u}(#1)}
\newcommand{\LISH}{{\circ}}
\newcommand{\lish}[1]{{#1}^{\LISH}}
\newcommand{\LOSH}{{\bullet}}
\newcommand{\losh}[1]{{#1}^{\LOSH}}

\newcommand{\todot}[1]{\todo[color=yellow!40, linecolor=orange!70!white]{\scriptsize #1}}
\newcommand{\todotp}[1]{\todo[prepend, color=yellow!40, linecolor=orange!70!white]{\scriptsize #1}}
\newcommand{\oqff}[2]{\textit{\textit{QF}}_{#1}({#2})}

\newcommand{\plx}{\textnormal{PLEXIL}}
\newcommand{\const}[1]{\mathit{#1}}

\newcommand{\lrcal}{\lish{\rcal}}
\newcommand{\purify}{\textit{abstract}}
\newcommand{\newvars}{\textit{fresh-vars}}

\newcommand{\id}[1]{\cde{#1}}
\newcommand{\cde}[1]{\ensuremath{\textit{#1}}}
\newcommand{\ded}{\vdash}

%% \newtheorem{definition}{Definition}
%% \newtheorem{lemma}{Lemma}
%% \newtheorem{theorem}{Theorem}
%% \newtheorem{corollary}{Corollary}
%% \newtheorem{remark}{Remark}

\lstdefinelanguage{Maude}{%
   keywords={
    , mod, fmod, endm, endfm
    , pr , protecting 
    , ex , extending 
    , inc, including
    , sort, sorts, subsort, subsorts
    , var, vars
    , op, ops
    , eq, ceq
    , rl, crl
    , if
    , search
    }
}
\lstnewenvironment{maude}
{\lstset{language=Maude ,
    keywordstyle=\color{blue}
  , basicstyle=\ttfamily\footnotesize %\singlespacing
  , commentstyle={}
  , columns=flexible
  , numbers=none
  , showstringspaces=false
  , keepspaces=true
  , escapechar=�
%  , aboveskip=-3pt
%  , belowskip=-3pt
  } 
}
{}

\DefineVerbatimEnvironment{maude2}{Verbatim}{fontsize=\scriptsize}
\newcommand{\rlname}[1]{\cde{[#1]}}

%% \theoremstyle{definition} \newtheorem{example}{Example}

\newcommand{\cesar}[2]{\sout{#1} {\textcolor{orange}{#2}}}%% CAM

\newcommand{\ccesar}[1]{\textbf{César:} {\textcolor{orange}{#1}}}%% CAM
\newcommand{\cjose}[1]{{\textcolor{blue}{#1}}} 
\newcommand{\ccamilo}[1]{{\textcolor{orange}{#1}}}
\newcommand{\cesarign}[2]{#2}
\newcommand{\joseign}[2]{#2}
\newcommand{\camiloign}[2]{#2}




\setbeamercolor{block title}{use=structure,fg=structure.fg,bg=structure.fg!20!bg}
\setbeamercolor{block body}{use=block title,bg=block title.bg!50!bg}


\title{Guarded Terms for Rewriting Modulo {SMT}}

\date{}

\author{\textbf{Kyungmin Bae} \and Camilo Rocha}

%\titlegraphic{\hfill\includegraphics[height=1.2cm]{icarous}}
\institute{
  14th International Conference on Formal Aspects of Component Software \\
  October 11, 2017 \\
  Braga, Portugal
}
% \titlegraphic{\hfill\includegraphics[height=1.5cm]{logo.pdf}}

\begin{document}

\maketitle

\begin{frame}{Context: Rewriting Modulo SMT}
\begin{outeritemize}
    \item \alert{Rewriting} Logic
    
    \begin{inneritemize}
        \item formalism to naturally specify concurrent/component systems
        \item actors, process calculi, AADL, Ptolemy~II, C, Java, \ldots
    \end{inneritemize}
    
    \pause
    \item \alert{Satisfiability Modulo Theories} (SMT)
    
    \begin{inneritemize}
        \item satisfiability of logic formulas with respect to decidable theories
        \item widely used for \cyan{symbolic analysis} of computer systems
    \end{inneritemize}
    
    \pause
    \item \colorbox{yellow}{Rewriting Modulo SMT}
    
    \begin{inneritemize}
        \item enable the \cyan{symbolic analysis} of many component systems
        \item e.g., PLEXIL by NASA for autonomous spacecraft operations
    \end{inneritemize}
\end{outeritemize}
\end{frame}



\begin{frame}{Rewriting Modulo SMT in a Nutshell (1)}
\begin{outeritemize}
    \item \alert{Symbolic state}: a term \blue{$t$} constrained by a logic formula \red{$\phi$}
    
    \begin{inneritemize}
        \item variables in \blue{$t$} ranging over the \alert{SMT built-ins}
        \item written as a pair $\pair{\blue{t}}{\red{\phi}}$
    \end{inneritemize}
    
    \pause
    \item Examples
    
    \begin{inneritemize}
        \item $(\blue{f(x, y)}; \red{x + y = 0})$
        \item $(\blue{x :: y :: z :: \mathit{nil}}; \red{x < y \,\wedge\, y < z})$ 
    \end{inneritemize}
    
    \pause
    \item Expresses a set of (potentially infinitely many) concrete
      states \hfill\purple{(written $\den{t}{\phi}$)}

    \begin{inneritemize}
        \item $f(0,0)$, $f(-1,1)$, $f(1,-1)$, $f(-2, 2)$, \ldots \hfill$\in \den{\blue{f(x, y)}}{\red{x + y = 0}}$
        \pause
        \item $0 :: 1 :: 2 :: \mathit{nil}$\;, $3 :: 5 :: 7 :: \mathit{nil}$, \ldots \hfill$\in \den{\blue{x :: y :: z :: \mathit{nil}}}{\red{x < y \wedge y < z}}$
    \end{inneritemize}
    
\end{outeritemize}
\end{frame}



\begin{frame}{Rewriting Modulo SMT in a Nutshell (2)}
\begin{outeritemize}
    \item \cyan{Conditional rewrite rules}: $\blue{t} \to \blue{t'} \;\;\mbox{\textbf{if}}\; \red{\varphi}$
    
    \begin{inneritemize}
        \item patterns $t_1$ and $t_2$, and SMT formula $\varphi$
        \item structural axioms (associativity, commutativity, identity)
    \end{inneritemize}
    
    \pause
    \item Rule $\blue{t} \to \blue{t'} \;\;\mbox{\textbf{if}}\; \red{\varphi}$ defines \alert{symbolic transition} 
    
    \begin{inneritemize}
        \item from symbolic state $\pair{t_1}{\phi_1}$ to $\pair{t_2}{\phi_2}$
        \item (infinitely many) instances of $t_1$ rewritten to instances of $t_2$
    \end{inneritemize}

    \pause
    \item Example:  $f(u,v) \to g(u) \;\;\mbox{\textbf{if}}\; {u > 0}$
    
    \begin{inneritemize}
        \pause
        \item $(\blue{f(x, y)}; \red{x + y = 0}) \longrightarrow (\blue{g(x)}; \red{x + y = 0 \,\wedge\, x > 0}) $
        \pause
        \item $(\blue{f(x - 1, 0)}; \red{x \leq 0}) \longrightarrow (\blue{g(x - 1)}; \red{x \leq 0 \,\wedge\, x - 1 > 0}) $
    \end{inneritemize}
\end{outeritemize}
\end{frame}



\begin{frame}{A Simple Example (1)}
\begin{outeritemize}
    \item Two components
    
    \begin{figure}
    \tikzstyle{comp}=[draw,text width=1.5cm, minimum height=1cm, rounded corners=0.1cm]
    \begin{tikzpicture}
    \node [comp] (c1) {$I_1, I_2, \ldots$};
    \path (c1) + (2.4cm,0) node [comp]  {$J_1, J_2, \ldots$};
    \path (c1) + (1.2cm,0) node {$\parallel$};
    \end{tikzpicture}
    \end{figure}
    
    \begin{inneritemize}
        \item each component has a collection of integers
        \item the left component sends each integer to the right
        \item the right component updates its state 
    \end{inneritemize}
    
    \pause
    \item Rules
    \[
    \scalebox{0.95}{
    \boxed{
    \begin{aligned}
    I_1, I_2, \mathit{C}_1 \parallel \mathit{C}_2 &\;\rews \;\;I_1,\mathit{C}_1 \parallel I_1 + I_2 + 1, \mathit{C}_2  && \textbf{if} \;\;   I_1 + I_2 \;\raisebox{-0.5ex}{\ensuremath{\stackrel{3}{\equiv}}}\; 0 \\
    I_1,I_2,\mathit{C}_1 \parallel \mathit{C}_2 &\;\rews\;\;I_1,\mathit{C}_1 \parallel \mathit{C}_2 \quad && \textbf{if} \;\;  I_1 + I_2 \;\raisebox{-0.5ex}{\ensuremath{\not\stackrel{3}{\equiv}}}\; 0 \\
    I_1 \parallel I_2,\mathit{C}_2 &\;\;\rews\;\;\textit{none} \parallel I_2,\mathit{C}_2 \quad && \textbf{if}\;\; I_1 + I_2 \;\raisebox{-0.5ex}{\ensuremath{\stackrel{17}{\equiv}}}\; 0
    \end{aligned}
    }
    }
    \]
\end{outeritemize}
\end{frame}



\begin{frame}{A Simple Example (1)}
\[
\scalebox{0.95}{
\boxed{
\begin{aligned}
I_1, I_2, \mathit{C}_1 \parallel \mathit{C}_2 &\;\rews \;\;I_1,\mathit{C}_1 \parallel I_1 + I_2 + 1, \mathit{C}_2  && \textbf{if} \;\;   I_1 + I_2 \;\raisebox{-0.5ex}{\ensuremath{\stackrel{3}{\equiv}}}\; 0 \\
I_1,I_2,\mathit{C}_1 \parallel \mathit{C}_2 &\;\rews\;\;I_1,\mathit{C}_1 \parallel \mathit{C}_2 \quad && \textbf{if} \;\;  I_1 + I_2 \;\raisebox{-0.5ex}{\ensuremath{\not\stackrel{3}{\equiv}}}\; 0 \\
I_1 \parallel I_2,\mathit{C}_2 &\;\;\rews\;\;\textit{none} \parallel I_2,\mathit{C}_2 \quad && \textbf{if}\;\; I_1 + I_2 \;\raisebox{-0.5ex}{\ensuremath{\stackrel{17}{\equiv}}}\; 0
\end{aligned}
}
}
\]
    
\begin{outeritemize}
    \item \alert{Concrete} execution
    \[
    \begin{aligned}
    1,2,3 \parallel \textit{none}
    \pause
    \;\longrightarrow\;
    1,3 \parallel 4
    \pause
    \;\longrightarrow\;
    1 \parallel 4
    \end{aligned}
    \]
    
    \pause
    \item \alert{Symbolic} execution \onslide<7->{\hfill\purple{(representing infinitely many paths)}}
    \[
    \begin{aligned}
    &
    ({\blue{I_1,I_2,I_3 \parallel \textit{none}}}; {\mathit{true}})
    \\
    \pause
    \longrightarrow\;&
    ({\blue{I_1,I_3 \parallel I_1 + I_2 + 1}}; {I_1 + I_2 \;\raisebox{-0.5ex}{\ensuremath{\stackrel{3}{\equiv}}}\; 0})
    \\
    \pause
    \longrightarrow\;&
    ({\blue{I_1 \parallel I_1 + I_2 + 1}}; {I_1 + I_2 \;\raisebox{-0.5ex}{\ensuremath{\stackrel{3}{\equiv}}}\; 0 \,\wedge\, I_1 + I_3 \;\raisebox{-0.5ex}{\ensuremath{\not\stackrel{3}{\equiv}}}\; 0})
    \end{aligned}
    \]

\end{outeritemize}
\end{frame}



\begin{frame}{A Simple Example (2)}
\[
\scalebox{0.95}{
\boxed{
\begin{aligned}
I_1, I_2, \mathit{C}_1 \parallel \mathit{C}_2 &\;\rews \;\;I_1,\mathit{C}_1 \parallel I_1 + I_2 + 1, \mathit{C}_2  && \textbf{if} \;\;   I_1 + I_2 \;\raisebox{-0.5ex}{\ensuremath{\stackrel{3}{\equiv}}}\; 0 \\
I_1,I_2,\mathit{C}_1 \parallel \mathit{C}_2 &\;\rews\;\;I_1,\mathit{C}_1 \parallel \mathit{C}_2 \quad && \textbf{if} \;\;  I_1 + I_2 \;\raisebox{-0.5ex}{\ensuremath{\not\stackrel{3}{\equiv}}}\; 0 \\
I_1 \parallel I_2,\mathit{C}_2 &\;\;\rews\;\;\textit{none} \parallel I_2,\mathit{C}_2 \quad && \textbf{if}\;\; I_1 + I_2 \;\raisebox{-0.5ex}{\ensuremath{\stackrel{17}{\equiv}}}\; 0
\end{aligned}
}
}
\]
    
\begin{outeritemize}
    \item \alert{Reachability property}: are there $I, \mathit{C}_1 \parallel \mathit{C}_2$ such that:
    
    \vspace{1ex}
    \begin{description}[initial pattern:]\small\addtolength{\itemsep}{0.2\baselineskip}
        \item[initial pattern:] \blue{$I, I+1, I+2, I+3 \parallel \textit{none}$}
        \item[goal:] \blue{$\mathit{C}_1 \parallel \mathit{C}_2$}, where $\mathit{C}_1$ is not empty
        $\begin{aligned}[t]
        \mbox{(i.e., the goal is a deadlock)}
        \end{aligned}$
    \end{description}

    \pause
    \item This query cannot be solved directly by rewriting
    
    \begin{inneritemize}
        \item because there are infinitely many initial states!!!
    \end{inneritemize}
\end{outeritemize}
\end{frame}



\begin{frame}[fragile]{A Simple Example (3)}
\begin{outeritemize}
    \item Such a query could be answered by rewriting modulo SMT 
    
    \item E.g., rewriting modulo SMT implementation in Maude:
    
    \begin{maude}
      search [1] { I, I+1, I+2, I+3 || none , I > 0 }
      =>! { C1 || C2, Phi } such that C1 =/= none .
    \end{maude}

    \pause
    \item However ...
\end{outeritemize}
\end{frame}



\begin{frame}{The Problem}
\begin{outeritemize}
    \item \ldots  \alert{scalability} is still a problem in
      rewriting modulo SMT, 
      
    \begin{inneritemize}
        \item some properties are beyond reach of rewriting modulo SMT
        \pause
        \item  as it is the case with most symbolic techniques
    \end{inneritemize}

    \pause
    \item Major reasons 
    
    \begin{inneritemize}
        \item the \alert{symbolic state space explosion problem}
        \item the \alert{complexity of the accumulated constraints}
    \end{inneritemize}

\end{outeritemize}
\end{frame}



\begin{frame}{Example: The Problem}
\[
\scalebox{0.95}{
\boxed{
\begin{aligned}
I_1, I_2, \mathit{C}_1 \parallel \mathit{C}_2 &\;\rews \;\;I_1,\mathit{C}_1 \parallel I_1 + I_2 + 1, \mathit{C}_2  && \textbf{if} \;\;   I_1 + I_2 \;\raisebox{-0.5ex}{\ensuremath{\stackrel{3}{\equiv}}}\; 0 \\
I_1,I_2,\mathit{C}_1 \parallel \mathit{C}_2 &\;\rews\;\;I_1,\mathit{C}_1 \parallel \mathit{C}_2 \quad && \textbf{if} \;\;  I_1 + I_2 \;\raisebox{-0.5ex}{\ensuremath{\not\stackrel{3}{\equiv}}}\; 0 \\
I_1 \parallel I_2,\mathit{C}_2 &\;\;\rews\;\;\textit{none} \parallel I_2,\mathit{C}_2 \quad && \textbf{if}\;\; I_1 + I_2 \;\raisebox{-0.5ex}{\ensuremath{\stackrel{17}{\equiv}}}\; 0
\end{aligned}
}
}
\]

\begin{outeritemize}
    \item For the initial pattern $I, (I+1), \ldots, (I+9) \parallel \textit{none}$
    \item How many reachable symbolic states?
    \item What are the SMT constraints?
\end{outeritemize}
\end{frame}



\begin{frame}{Main Contribution and Approach}
\begin{center}\textbf{Guarded Terms for Rewriting Modulo SMT}\end{center}
\begin{outeritemize}
    \item Goal
    
    \begin{inneritemize}
        \item reduce the symbolic state space 
        \item decrease the complexity of constraints
    \end{inneritemize}
    
    \item Guarded term 
    
    \begin{inneritemize}
        \item generalizes a constrained term in rewriting modulo SMT 
        \item encode branching by nondeterminism as part of term structure
    \end{inneritemize}

%      \item Intuitively, they can be seen as a choice operator that is
%        part of the term structure in a constrained term.
\end{outeritemize}
\end{frame}



\begin{frame}{Guarded Terms: Parallel Composition}
\begin{outeritemize}
    \item Combining constrained terms $\pair{t_1}{\phi_1},\ldots,\pair{t_n}{\phi_n}$ in \alert{parallel}
    \[\pair{t_1}{\phi_1} \lor \cdots \lor \pair{t_n}{\phi_n}\]
    
    \item Semantically representing the \cyan{union} of their instance sets
    \[\den{t_1}{\phi_1} \cup \cdots \cup \den{t_n}{\phi_n}\]
\end{outeritemize}

%  \[\den{t}{\phi} = \left\{t' \in T_{\Sigma,\sort{Conf}} \mid \;
%(\exists \func{\sigma}{X}{T_\Sigma})\, t' = \sigma t \land
%  \tcal_{\Sigma/E\uplus B} \models \sigma\phi \right\}.\]
\end{frame}



\begin{frame}{Guarded Terms: Nested Composition}
\begin{outeritemize}
    \item Guarded terms can be \alert{nested} as subterms, e.g., 
    \[f(\pair{t_1}{\phi_1} \lor \pair{f(g(\pair{t_3}{\phi_3},t_4))}{\phi_2})\]
    
    \item Semantically representing all corresponding instances
    \[\den{f(\pair{u_1}{\phi_1},\ldots,\pair{u_n}{\phi_n})}{} = \bigcup_{t_i \in \den{u_i}{\phi_i}}\{f(t_1,\ldots,t_n)\} \]
    
\end{outeritemize}
\end{frame}



\begin{frame}{Power of Guarded Terms (1)}
\begin{outeritemize}
    \item A guarded term can express \alert{exponentially many}
      symbolic states
    
    \begin{inneritemize}
        \item that is, symbolic state space can be exponentially
          smaller
    \end{inneritemize}
    
    \pause
    \item Example: $\pair{f(u_1 \vee v_1, u_2 \vee v_2, u_3 \vee v_3)}{\phi}$
    
    \begin{inneritemize}
        \item $\pair{f(u_1, u_2, u_3)}{\phi}$, $\pair{f(u_1, u_2, v_3)}{\phi}$, $\pair{f(u_1, v_2, u_3)}{\phi}$, $\pair{f(u_1, v_2, v_3)}{\phi}$, $\pair{f(v_1, u_2, u_3)}{\phi}$, $\pair{f(v_1, u_2, v_3)}{\phi}$, $\pair{f(v_1, v_2, u_3)}{\phi}$, $\pair{f(v_1, v_2, v_3)}{\phi}$
    \end{inneritemize}

\pause
\vspace{1em}
\begin{theorem}\small
\label{thm:standard}
Every guarded term is equivalent to guarded term in \alert{standard form} 
\[
\pair{t_1}{\phi_1} \lor \pair{t_2}{\phi_2} \lor \cdots \lor \pair{t_n}{\phi_n}\]
\end{theorem}
\end{outeritemize}
\end{frame}



\begin{frame}{Power of Guarded Terms (2)}
\begin{outeritemize}
    \item A set of rewrite rules can be combined using guarded terms
    
    \begin{inneritemize}
        \item so that \alert{branching by nondeterminism is removed}
    \end{inneritemize}
    
    \pause
    \item Example
    \[
    \scalebox{0.9}{\ensuremath{
    \begin{gathered}
    \boxed{
    \begin{aligned}
    I_1, I_2, \mathit{C}_1 \parallel \mathit{C}_2 &\;\rews \;\;I_1,\mathit{C}_1 \parallel I_1 + I_2 + 1, \mathit{C}_2  && \textbf{if} \;\;   I_1 + I_2 \;\raisebox{-0.5ex}{\ensuremath{\stackrel{3}{\equiv}}}\; 0 \\
    I_1,I_2,\mathit{C}_1 \parallel \mathit{C}_2 &\;\rews\;\;I_1,\mathit{C}_1 \parallel \mathit{C}_2 \quad && \textbf{if} \;\;  I_1 + I_2 \;\raisebox{-0.5ex}{\ensuremath{\not\stackrel{3}{\equiv}}}\; 0 
    %\\
    %I_1 \parallel I_2,\mathit{C}_2 &\;\;\rews\;\;\textit{none} \parallel I_2,\mathit{C}_2 \quad && \textbf{if}\;\; I_1 + I_2 \;\raisebox{-0.5ex}{\ensuremath{\stackrel{17}{\equiv}}}\; 0
    \end{aligned}
    }
    \\[-0.5ex]
    \Downarrow
    \\[-0.5ex]
    \boxed{
    \begin{aligned}
    I_1, I_2, \mathit{C}_1 \parallel \mathit{C}_2 &\;\rews \;\;I_1, \mathit{C}_1 \parallel \begin{multlined}[t](I_1 + I_2 + 1; (I_1 + I_2) \;\raisebox{-0.5ex}{\ensuremath{\stackrel{3}{\equiv}}}\; 0)\\ \vee (\mathit{none}; (I_1 + I_2) \;\raisebox{-0.5ex}{\ensuremath{\not\stackrel{3}{\equiv}}}\; 0)\;\mathit{C}_2\end{multlined} 
    \end{aligned}
    }
    \end{gathered}
    }}
    \]
\end{outeritemize}

\pause
\begin{theorem}\small
\label{thm:sim}
A set $R$ of guarded rewrite rules is equivalent to its \alert{standard expansion}
(in terms of a simulation relation).
\end{theorem}
\end{frame}



\begin{frame}[fragile]{The Running Example with Guarded Terms}
\begin{outeritemize}
    \item \alert{Previously timed-out}, but now proven using guarded terms
    
\begin{maude}
search [1] { I, I+1, I+2, I+3 || none , I > 0 }
       =>! { C1 || C2, Phi } such that C1 =/= none .
No solution.
states: 265  rewrites: 37516 in 702ms cpu (704ms real) 
\end{maude}
\end{outeritemize}
\end{frame}



\begin{frame}{The CASH Algorithm: a More Interesting Case Study (1)}
\begin{outeritemize}
    \item A real-time scheduling algorithm with capacity sharing
    
    \begin{inneritemize}
        \item for reusing unused execution budgets
        \item make use of \cyan{unbounded priority queue}
    \end{inneritemize}

    \item It has been formally specified and analyzed in:
    
    \begin{inneritemize}
        \item Real-Time Maude using explicit-state methods %(ground rewriting)
        \item Rewriting modulo SMT to deal with infinite number of states
    \end{inneritemize}
    
    \pause
    \item Reachability property 
    
    \begin{inneritemize}
        \item the existence of missed deadlines in the variant of CASH
        \item when the processor utilization $\leq 100 \%$
    \end{inneritemize}
    
\end{outeritemize}
\end{frame}



\begin{frame}{The CASH Algorithm: a More Interesting Case Study (2)}
\begin{outeritemize}

    \item The property could not be analyzed by rewriting modulo SMT
    
    \begin{inneritemize}
        \item due to the \alert{symbolic state space explosion problem}
        \item generated more than one billion \cyan{symbolic states}
    \end{inneritemize}
    
    \pause
    \item Coping with the symbolic state space explosion
    
    \begin{inneritemize}
        \item remove symbolic branching using \alert{guarded terms}
        \item ``combine'' different rewrite rules using guarded terms
    \end{inneritemize}

    \pause
    \item The same property can be automatically proved
    
    \begin{inneritemize}
        \item counterexample found (at search depth 23)
        \item after 600,000 symbolic states explored (about 90 min)
    \end{inneritemize}    
\end{outeritemize}
\end{frame}



\begin{frame}{Concluding Remarks}
\begin{outeritemize}
    \item \blue{Rewriting modulo SMT}
    
    \begin{inneritemize}
        \item enable the symbolic analysis of many component systems
        \item the \alert{symbolic state space explosion problem}
    \end{inneritemize}   
    
    \item \colorbox{yellow}{Guarded terms}
      
    \begin{inneritemize}
        \item generalize the constrained terms of rewriting modulo SMT 
        \item reduce symbolic state space and complexity of constraints
    \end{inneritemize}

    \item The CASH case study
    
    \begin{inneritemize}
        \item nontrivial property was beyond  reach of rewriting modulo SMT        
        \item but can now be proved using guarded terms 
    \end{inneritemize}

    \item Future work 
    
    \begin{inneritemize}
        \item explore new uses of guarded terms on more case studies
        \item combination with other state reduction techniques
    \end{inneritemize}
  \end{outeritemize}
\end{frame}



\begin{frame}[standout]
  Questions?
\end{frame}

\end{document}
